% Metódy inžinierskej práce

\documentclass[10pt,twoside,english,a4paper]{article}

\usepackage[english]{babel}
%\usepackage[T1]{fontenc}
\usepackage[IL2]{fontenc} % lepšia sadzba písmena Ľ než v T1
\usepackage[utf8]{inputenc}
\usepackage{graphicx}
\usepackage{url} % príkaz \url na formátovanie URL
\usepackage{hyperref} % odkazy v texte budú aktívne (pri niektorých triedach dokumentov spôsobuje posun textu)

\usepackage{cite}
%\usepackage{times}

\pagestyle{headings}

\title{Problems with eLearning associated with distinctive learning styles of students\thanks{Semestrálny projekt v predmete Metódy inžinierskej práce, ak. rok 2020/21, vedenie: Meno Priezvisko}} % meno a priezvisko vyučujúceho na cvičeniach

\author{Adrián Topolski\\[2pt]
	{\small Slovenská technická univerzita v Bratislave}\\
	{\small Fakulta informatiky a informačných technológií}\\
	{\small \texttt{xtopolski@stuba.sk}}
	}

\date{\small 27. October 2020} % upravte



\begin{document}

\maketitle

\begin{abstract}
The knowledge of a preson's learning style is crucial for maximizing the effectiveness of learning. During their practise of tradicional teaching, many teachers and professors have developed their own way of creating the best way of interpretation to suit all students, regardless of learning style. However, with the sudden need to transfer to eLearning, teachers weren't able to use their previous ways of teaching to full extent, often leaving gaps in educating students with certain learning styles. Such unprofessional eLearning style can significantly elongate the time it takes for an individual to comprehend topic that the teacher is explaining and may even cause sudden decline in grades, only because the current way of teaching has stopped taking into account the learning style of the student. However, even though this eLearning environment might never be as efficient for some people as tradicional education, as always, there are ways of mitigating these problems. 
\ldots
\end{abstract}



\section{Úvod} 



Motivujte čitateľa a vysvetlite, o čom píšete. Úvod sa väčšinou nedelí na časti.

Uveďte explicitne štruktúru článku. Tu je nejaký príklad.
Základný problém, ktorý bol naznačený v úvode, je podrobnejšie vysvetlený v časti~\ref{FS}.
Dôležité súvislosti sú uvedené v častiach~\ref{dolezita} a~\ref{dolezitejsia}.
Záverečné poznámky prináša časť~\ref{zaver}.



\section{Felder-Silverman Learner Style Model} \label{FS}
In 1988 Richard M. Felder and Linda K. Silverman developed a learning style model known as Felder Silverman Learner Style Model. The point of this model was to focus on engineering students and discover their learning styles.\cite{AdaptiveEL} According to answers to five questions, they divided learning styles of students to five dimensions:
\begin{enumerate}
\item Perception
\item Input
\item Organization
\item Processing
\item Understanding
\end{enumerate}\cite{FelderArticle}
{\subsection{Perception}
This category originates from Carl Jung's theory of psychological types. He stated that sensing and intuition are the two ways people tend to perceive the world. Sensory and intuitive learners fall into this category. While sensory learners gather data by observing and using their senses, intuitive learners use their imagination and speculation. While no one is exclusively neither intuitive nor sensory, people tend to prefer one over the other.\cite{FelderArticle}}
{\subsection{Input}
Input defines the way how people perceive information. It can be divided into visual and verbal learners. Visual learners prefer pictures, diagrams, demonstrations, etc. while verbal learners prefer spoken and written words. In has been established that people tend to ignore information presented to them in the way that they do not prefer. For example, visual learners tend to often forget what they have heard while verbal learners get a lot out of a discussion.\cite{FelderArticle}}
{\subsection{Organization}
This category classifies people into inductive and deductive learners. Inductive learners learn by induction, therefore by inferring principles, whereas deductive learners deduce consequences.\cite{FelderArticle}}
{\subsection{Processing}
Processing takes into account the conversion of perceived information into knowledge. In this category, people are classified as active or reflective learners. Active learners learn best by manipulating with the information via active experimentation or discussion. Reflective learners need time to think about the presented information and examine it introspectively. It is also stated that active learners work well in groups whereas reflective learners work best alone or in groups of two.\cite{FelderArticle}}
{\subsection{Understanding}
Understanding states student's progression toward understanding. Sequential learners learn best by learning material in a logically ordered progression. When solving problems, they use linear reasoning. Global learners learn best by exploring larger dimensions of information and make intuitive leaps. Because most formal education presents material in a logically ordered progression, school is often a difficult experience for them.\cite{FelderArticle, AdaptiveEL}}


They also classified teaching styles corresponding to the individual learning styles of students. Teaching styles can be classified as well by the answers to five questions. These dimensions of teaching styles are:
\begin{enumerate}
\item Content
\item Presentation
\item Organization
\item Student participation
\item Perspective
\end{enumerate}

\subsection{Ja ti dám nejaká časť!}
Z obr.~\ref{f:rozhod} je všetko jasné. 

\begin{figure*}[tbh]
\centering
%\includegraphics[scale=1.0]{diagram.pdf}
Aj text môže byť prezentovaný ako obrázok. Stane sa z neho označný plávajúci objekt. Po vytvorení diagramu zrušte znak \texttt{\%} pred príkazom \verb|\includegraphics| označte tento riadok ako komentár (tiež pomocou znaku \texttt{\%}).
\caption{Rozhodujúci argument.}
\label{f:rozhod}
\end{figure*}



\section{Iná časť} \label{ina}

Základným problémom je teda\ldots{} Najprv sa pozrieme na nejaké vysvetlenie (časť~\ref{ina:nejake}), a potom na ešte nejaké (časť~\ref{ina:nejake}).\footnote{Niekedy môžete potrebovať aj poznámku pod čiarou.}

Môže sa zdať, že problém vlastne nejestvuje\cite{Coplien:MPD}, ale bolo dokázané, že to tak nie je~\cite{Czarnecki:Staged, Czarnecki:Progress}. Napriek tomu, aj dnes na webe narazíme na všelijaké pochybné názory\cite{PLP-Framework}. Dôležité veci možno \emph{zdôrazniť kurzívou}.


\subsection{Nejaké vysvetlenie} \label{ina:nejake}

Niekedy treba uviesť zoznam:

\begin{itemize}
\item jedna vec
\item druhá vec
	\begin{itemize}
	\item x
	\item y
	\end{itemize}
\end{itemize}

Ten istý zoznam, len číslovaný:

\begin{enumerate}
\item jedna vec
\item druhá vec
	\begin{enumerate}
	\item x
	\item y
	\end{enumerate}
\end{enumerate}


\subsection{Ešte nejaké vysvetlenie} \label{ina:este}

\paragraph{Veľmi dôležitá poznámka.}
Niekedy je potrebné nadpisom označiť odsek. Text pokračuje hneď za nadpisom.



\section{Dôležitá časť} \label{dolezita}




\section{Ešte dôležitejšia časť} \label{dolezitejsia}




\section{Záver} \label{zaver} % prípadne iný variant názvu



%\acknowledgement{Ak niekomu chcete poďakovať\ldots}

\bibliographystyle{plain} % prípadne alpha, abbrv alebo hociktorý iný
% týmto sa generuje zoznam literatúry z obsahu súboru literatura.bib podľa toho, na čo sa v článku odkazujete
\bibliography{zdroje}


\end{document}
